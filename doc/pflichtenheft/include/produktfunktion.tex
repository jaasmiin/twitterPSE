% Detailliertere  Beschreibung  der  Funktionalität,  wiederum  gegliedert  in Grundfunktionen und optionale Funktionen. 

Die Beschreibung der Funktionalität gliedert sich in Grundfunktionen und optionale Funktionen.

\section{Grundfunktionen}

Die Grundfunktionen der Anwendung gliedern sich in verschiedene Bereiche.

%\subsection{Allgemeine Funktionalität}
%\begin{enumerate}[align=left, label={\textbf{\textbackslash F00\arabic*0\textbackslash}} ]
%	\item Start des Programms \\
%	Bei Start des Programms wird die leere Weltkarte angezeigt.
%	\begin{itemize}
%		\item Hier könnte auch Ihre Werbung stehen ...
%	\end{itemize}


%\end{enumerate}
\subsection{Auswahl der gewünschten Daten} \label{sec:Produktfunktionen}
\begin{enumerate}[ align=left, label={\textbf{\textbackslash F10\arabic*0\textbackslash}} ]
	\item \textbf{Auswahl der Kategorien} \label{PF:KategorienAuswahl}\\ 
	Die verschiedenen Kategorien werden angezeigt. Man kann durch die Auswahl navigieren und eine oder mehrere Kategorien auswählen. Hierfür kann eine Suchfunktion genutzt werden ($\rightarrow$ \ref{PF:Suchfunktion}). 
	\item \textbf{Auswahl der Ortsbeschränkung} \label{PF:OrtAuswahl} \\
	Die verschiedenen Orte werden angezeigt. Man kann durch die Auswahl navigieren und einen oder mehrere Orte auswählen. Hierfür kann eine Suchfunktion genutzt werden ($\rightarrow$ \ref{PF:Suchfunktion}). 
	\item \textbf{Manuelles Hinzufügen weiterer Accounts} \label{PF:AccountHinzufügen} \\
	Mittels der Suchfunktion ($\rightarrow$ \ref{PF:Suchfunktion}) können weitere Accounts einzeln zur Auswahl hinzugefügt werden, sofern diese in der Datenbank hinterlegt sind.
	\item \textbf{Suchfunktion innerhalb des Auswahlprozesses} \label{PF:Suchfunktion} \\
	Es kann nach Kategorien, Orten und Benutzern gesucht werden, die bereits in der Datenbank vorhanden sind. Die Funktionalität unterscheidet sich je nach der Art des gesuchten Objekts. In allen Fällen wird eine Liste mit Vorschlägen abhängig vom gerade eingegebenen Suchbegriff angezeigt. 
	\item \textbf{Aktualisierung der Anfrage} \label{PF:Absenden} \\
	Bei Änderung der Anfrageparameter werden die Daten automatisch ausgewertet und nach einer gewissen Latenzzeit angezeigt.
\end{enumerate}	
\subsection{Visualisierung der Daten}	
Für die Visualisierung der Daten bestehen folgende Funktionen.
\begin{enumerate}[ align=left, label={\textbf{\textbackslash F20\arabic*0\textbackslash}} ]	
	\item \textbf{Anzeige der grafisch aufbereiteten Daten} \label{PF:Visualisieren}\\
	Die Daten der aktuellen Anfrage werden grafisch aufbereitet in der Karte dargestellt.
	\begin{itemize}
		\item Einfärben der Karte ($\rightarrow$ \ref{PF:EinfärbenKarte}).
		\item Detaillierte Informationen bezüglich einzelner Orte ($\rightarrow$ \ref{PF:DetailsKarte}).
	%	\item Wahl des Beobachtungszeitraums ($\rightarrow$ \ref{PF:WahlZeitraum}).
	%	\item Anzeige von Veränderungen des Tweet-Retweet-Verhaltens über einen Zeitraum ($\rightarrow$ \ref{PF:Diff})
	\end{itemize}
	
	\item \textbf{Einfärben der Karte} \label{PF:EinfärbenKarte} \\
	Die Länder/Regionen werden auf der Karte bezüglich der Tweet-Retweet-Intensität unterschiedlich eingefärbt. Hierbei handelt es sich um relative Daten, um Bevölkerungsunterschiede auszugleichen.
	\item \textbf{Detaillierte Informationen bezüglich einzelner Orte} \label{PF:DetailsKarte} \\
	Beim Bewegen des Mauszeigers über einen Ort werden detaillierte Informationen wie Anzahl der Retweets über die Tweet-Retweet-Beziehung der aktuellen Anfrage an diesem Ort angezeigt. Diese Option kann deaktiviert werden.
	
	
	\item \textbf{Anzeige mittels Datenblatt} \label{PF:AnzeigeDatenblatt} \\
	Die relevanten Daten der Anfrage werden in Tabellenform dargestellt. Zu einem Ort-Kategorie-Paar können die zugehörigen Accounts aufgelistet werden.
	

\end{enumerate}

\subsection{Navigation auf der Weltkarte}
\begin{enumerate}[ align=left, label={\textbf{\textbackslash F30\arabic*0\textbackslash}} ]
	\item \textbf{Navigation auf der Karte} \label{PF:Navigation} \\
	Es ist möglich die Karte zu verschieben.
	\item \textbf{Zoomen} \label{PF:Zoomen} \\
	Die Ansicht auf die Karte kann, innherhalb gewisser Grenzen, vergrößert und verkleinert werden.
\end{enumerate}	
\subsection{Manuelle Bearbeitung der Datenbank}
Durch den Benutzer können einige Änderungen der in der Datenbank  gespeicherten Daten vorgenommen werden. Diese Funktionen müssen gegebenenfalls eingeschränkt werden, sollte die Anwendung öffentlich verwendet werden.
\begin{enumerate}[ align=left, label={\textbf{\textbackslash F40\arabic*0\textbackslash}}]
	\item \textbf{Kategorie eines Accounts ändern}  \label{PF:KategorieAendern} \\
	Die Kategorie eines gespeicherten Accounts kann in eine schon in der Datenbank vorhandene Kategorie geändert werden.
	\item \textbf{Ort eines Accounts ändern} \label{PF:OrtAendern}\\
	Der Ort eines Account kann in einen schon in der Datenbank vorhandenen Ort geändert werden.
	\item \textbf{Account hinzufügen} \label{PF:AccountHinzu} \\
	Es kann ein Twitter-Account, sofern dieser existiert, zu der Liste der mitgelesenen Twitter-Accounts hinzugefügt werden, auch wenn dieser nicht verifiziert ist.
\end{enumerate}

\section{Optionale Funktionalität}
Die optionalen Funktionen sind nach absteigender Priorität aufgelistet.
\begin{enumerate}[ align=left, label={\textbf{\textbackslash F50\arabic*0\textbackslash}}]
	\item \textbf{Anzeigen von Veränderungen  zwischen zwei Zeitpunkten} \label{PF:Diff} \\
	Die Unterschiede in der Tweet-Retweet-Beziehung können zwischen zwei, innerhalb gewisser Grenzen, wählbaren Zeitpunkten grafisch dargestellt werden.
	\item \textbf{Wahl des Beobachtungszeitraums} \label{PF:WahlZeitraum} \\
	Der Zeitraum, für welchen die Tweet-Retweet-Beziehung dargestellt werden soll, kann innerhalb gewisser Grenzen und mit diskreten Zeitpunkten ausgewählt werden.
	\item \textbf{Allgemeine Suchfunktion} \label{PF:AllgSuche} \\
	Es wird eine allgemeine Suchfunktion angeboten. Hier kann nach beliebigen Begriffen gesucht werden. Die  Daten, welche zum jeweiligen Begriff in der Datenbank verfügbar sind, werden angezeigt.
	\item  \textbf{Sicherung von Ergebnissen} \label{PF:Sicherung} \\
	Die zu einer Anfrage erstellten Datenblätter und Kartenanzeigen können in geeignete Dateiformate exportiert und somit dauerhaft gespeichert werden.
	\item \textbf{Feingliederung der Auswahlmöglichkeiten} \\
	Es ist innerhalb einer größeren Anfrage möglich, einzelne Kategorien auf eine Teilmenge der insgesamt ausgewählten Orte zu beschränken. Man kann somit beispielsweise die Anfrage nach der Kategorie \emph{Musiker} auf den Ort \emph{Irland} einschränken und  derselben Anfrage noch die Kategorie \emph{Politiker} eingeschränkt auf den Ort \emph{Deutschland} hinzufügen.
	\item \textbf{Verknüpfung mit Twitteraccount} \label{PF:Verknuepfung} \\
	Ein Account in der Datenbank wird mit seinem Twitterprofil verknüpft. Wird der Account beispielsweise über die allgemeine Suchfunktion ($\rightarrow$ \ref{PF:AllgSuche}) gesucht und aufgerufen, werden Informationen aus der Timeline des Accounts angezeigt. 
	\item \textbf{Weitere Informationen zu Kategorien und Orten} \label{PF:WeiterInfos} \\
	Wird ein Ort oder eine Kategorie über die Suchfunktion ($\rightarrow$ \ref{PF:AllgSuche}) gesucht und können zusätzlich Informationen, die beispielsweise von wikipedia.org geladen werden, angezeigt.
	
	\item \textbf{Weitere Statistiken} \label{PF:Statistiken} \\
	Es werden noch weitere Statistikfunktionen und Anzeigeoptionen angeboten.
\end{enumerate}
