% Esentielle Begriffe.

\begin{description}
	\item[Account] Profil eines beim Webdienst Twitter angemeldeten Nutzers.Synonyme: User.
	\item[Client] Synonyme Verwendung für die Hardware auf welcher die GUI läuft als auch für die Software der GUI selbst.
	\item[Crawler] Systembestandteil, welches auf dem Server läuft und die Daten von Twitter sammelt, lokalisiert und in die Datenbank schreibt.
	\item [DMOZ-Datenbank] Open-Directory Datenbank, welche Begriffe und Personen thematisch in Kategorien unterteilt.
	\item[GUI] Graphical-User-Interface (Benutzerschnittstelle).
	\item[Kategorie] Thematische Untergliederung von Begriffen und Personen in Kategorien nach der DMOZ.org Datenbank.
	\item[Kategorisierer] Systembestandteil, welches auf dem Server läuft und die Accounts in der Datenbank mithilfe der DMOZ.org Datenbank kategorisiert.
	\item[Retweet] Ein Retweet ist die Weiterverbreitung eines abgesendeten Tweets.
	\item[Server] Hardware, auf der Crawler, Kategorisierer und  Datenbank laufen. Auch Sammelbegriff für die Systembestandteile Crawler, Kategorisierer und Datenbank.
	\item[Tweet] Ein Beitrag eines Twitternutzers.
	\item[Tweet-Retweet-Beziehung] Im Kontext dieser Anwendung ist hiermit die Anzahl der Retweets auf Tweets eines Accounts gemeint. 
	\item[verifizierter Account] Ein Twitter-Account, welcher durch Twitter verifiziert wurde. Die Person, Firma oder Organisation, die den Account augenscheinlich betreibt, ist der tatsächliche Inhaber des Accounts. Es handelt sich also um keinen Fake-Account.
\end{description}
