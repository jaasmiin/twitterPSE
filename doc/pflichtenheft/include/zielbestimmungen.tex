% Beschreibt die Funktionalität der zu entwickelnden Systemkomponente. 
% o Musskriterien: Mindestanforderungen, gehen aus Aufgabenstellung hervor. 
% o Wunschkriterien: Von den Gruppen selbst definierte, zusätzliche Funktionalität. 
% o Abgrenzungskriterien: Was gehört nicht zum Funktionsumfang? (vgl. auch Produkteinsatz) 

\section{Musskriterien}
\begin{itemize}
	\item Es soll ein Programm entwickelt werden, dass Twitterdaten (mittels Streaming- und Search-API) nach verifizierten Nutzern durchsucht und diese speichert.
	\item Zu Tweets von verifizierten Accounts sollen Retweets geografisch gruppiert und gezählt werden.
	\item Dazu soll einzelnen Accounts eine geografische Position und eine thematische Kategorie zugeordnet werden.
	\item Es sollen interaktiv Anfragen an das System gestellt werden können: Anfragen bestehen zum Beispiel aus einer Kategorie und einem Land. Damit soll die weltweite geografische Ausbreitung von Tweets mit diesen Eigenschaften grafisch dargestellt werden (siehe \cref{sec:Produktfunktionen}).
	\item Die Interaktion mit dem System soll über eine intuitiv bedienbare grafische Benutzeroberfläche geschehen. Diese soll neben den Auswahlhilfen eine Weltkarte sowie weitere Statistiken zeigen.
	\item Über die Benutzeroberfläche sollen Accounts manuell kategorisiert und lokalisiert werden können.
\end{itemize}

\section{Wunschkriterien}
Folgende Auflistung zeigt die Wunschkriterien in absteigender Priorität.
\begin{enumerate}
	\item Es soll möglich sein, nicht verifizierte Benutzer einzugeben, sodass die Ausbreitung ihrer Tweets analysiert werden kann.
	\item Visualisierung von Datenströmen auf der Karte.
	\item Eine Visualisierung der Analyseergebnisse mit Schwerpunkt auf der zeitlichen Entwicklung sowie weiterer Statistiken wie der Häufigkeit bestimmter Kategorien in verschiedenen Ländern.
	\item Eine hierarchische Einteilung der Regionen in Ergänzung zur Analyse pro Land. So könnte zum Beispiel pro Kontinent  visualisiert werden.
	\item Exportfunktion für Analyseergebnisse.
\end{enumerate}

\section{Abgrenzungskriterien}
\begin{itemize}
	\item Es werden Retweets pro Account, nicht pro Tweet analysiert. Daher wird es nicht möglich sein, einzelne Tweets zu analysieren.
	\item Es handelt sich um ein Analyseprogramm, nicht um ein Lokalisierungsprogramm. Zur Lokalisierung soll ein existierender Webservice genutzt werden.
	\item Die Analyse beruht nicht auf dem Inhalt der Tweets (Hashtags, etc.) sondern ausschließlich auf der dem Account zugeordneten Kategorie und Ort.
\end{itemize}
