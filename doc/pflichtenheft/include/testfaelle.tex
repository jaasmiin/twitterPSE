% Testfälle  für  die  einzelnen  Produktfunktionen,  die  alle  abgedeckt  sein müssen. Testszenarien für typische Anwendungsszenarien.

\section{Testszenarien}
\begin{itemize}
	\item \textbf{„Programm starten und Twitterdatenströme anzeigen lassen“}\\
Programm starten $\rightarrow$ Häkchen im Kategorienbaum setzen $\rightarrow$ Land auswählen $\rightarrow$ Start-Knopf drücken $\rightarrow$ Weltkarte anzeigen lassen.

	\item \textbf{„Bestimmten verifizierten User finden “}\\
Programm starten $\rightarrow$ In Suchfeld Accountnamen einfügen $\rightarrow$ Auf „Suchen“ klicken $\rightarrow$ Falls Account noch nicht vorhanden ist hinzufügen $\rightarrow$ Ansonsten auf „OK“ drücken $\rightarrow$ Weltkarte anzeigen lassen.

	\item \textbf{„Bei bestehendem Account Kategorien bearbeiten“}\\
In Suchleiste Accountnamen eintippen $\rightarrow$ Sobald Account gefunden wurde auf „Bearbeiten“ drücken $\rightarrow$ Kategorien und/oder Ortsangaben ändern $\rightarrow$ Auf „Speichern“ klicken.

	\item \textbf{„Neuen Account zum crawlen hinzufügen“}\\
Auf „Account hinzufügen“ klicken $\rightarrow$ Accountnamen und ggf. andere Informationen zum Account eingeben $\rightarrow$ Auf „Speichern“ klicken.

	\item \textbf{„Kategorie umbenennen“}\\
Programm starten $\rightarrow$ Gewünschte Kategorie suchen $\rightarrow$ Auf „Bearbeiten“  klicken $\rightarrow$ Kategorienamen umbenennen $\rightarrow$ Auf „Speichern“ klicken.

	\item \textbf{„Differenz zwischen zwei Zeiträumen anzeigen lassen“}\\
Programm starten $\rightarrow$ Zeitdifferenzfunktion aufrufen $\rightarrow$ Startdatum angeben $\rightarrow$ Enddatum angeben $\rightarrow$ “Speichern“ anklicken $\rightarrow$ Zeitliche Differenz auf Weltkarte anzeigen lassen.


\end{itemize}