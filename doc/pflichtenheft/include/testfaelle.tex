% Testfälle  für  die  einzelnen  Produktfunktionen,  die  alle  abgedeckt  sein müssen. Testszenarien für typische Anwendungsszenarien.

\section{Testfälle}

Die Testfälle gliedern sich in allgemeine Testfälle für die GUI und Testfälle für die Datenbank.

	\subsection{GUI Tests}
	\begin{enumerate}[align=left, label={\textbf{\textbackslash T10\arabic*0\textbackslash}} ]
	\item Start des Programms
	\item Kategorie auswählen
	\item Ort auswählen
	\item Nach Kategorie suchen
	\item Nach Ort suchen
	\item Nach Benutzer suchen
	\item Hinzufügen eines neuen Accounts
	\item Beobachtungszeitraum wählen
	\item Weltkarte anzeigen lassen
	\item Daten in Tabelle anzeigen lassen
	\item Beobachtungszeitraum bearbeiten
	\item Auf Karte zoomen
	\end{enumerate}
	
	\subsection{Datenbank}
	\begin{enumerate}[align=left, label={\textbf{\textbackslash T20\arabic*0\textbackslash}} ]
		\item Kategorie umbenennen
		\item Neue Kategorie erstellen und hinzufügen
		\item Neuen Account hinzufügen
		\item Kategorie eines Accounts ändern/löschen/hinzufügen
		\item Ort eines Accounts ändern
	\end{enumerate}

\section{Testszenarien}
\begin{itemize}
	\item \textbf{„Programm starten und Twitterdatenströme anzeigen lassen“}\\
Programm starten $\rightarrow$ Kategorie auswählen $\rightarrow$ Ort auswählen $\rightarrow$ Weltkarte anzeigen lassen.

	\item \textbf{„Bestimmten verifizierten User finden “}\\
Programm starten $\rightarrow$ In Suchfeld Accountnamen einfügen $\rightarrow$ Nach gewünschtem Benutzer suchen $\rightarrow$ Falls Account noch nicht vorhanden ist, neuen Benutzer hinzufügen $\rightarrow$ Ansonsten Benutzer auswählen $\rightarrow$ Weltkarte anzeigen lassen.

	\item \textbf{„Bei bestehendem Account Kategorien bearbeiten“}\\
Programm starten $\rightarrow$ Nach Benutzer suchen $\rightarrow$ Sobald Account gefunden wurde auf „Bearbeiten“ drücken $\rightarrow$ Kategorien und/oder Ortsangaben ändern $\rightarrow$ Auf „Speichern“ klicken.

	\item \textbf{„Neuen Account zum crawlen hinzufügen“}\\
Programm starten $\rightarrow$ Auf „Hinzufügen“ klicken $\rightarrow$ Auf „Benutzer“ klicken $\rightarrow$ Accountnamen und ggf. andere Informationen zum Account eingeben $\rightarrow$ Auf „Speichern“ klicken.

	\item \textbf{„Kategorie umbenennen“}\\
Programm starten $\rightarrow$ Gewünschte Kategorie suchen $\rightarrow$ Auf „Bearbeiten“  klicken $\rightarrow$ Kategorienamen umbenennen $\rightarrow$ Auf „Speichern“ klicken.

	\item \textbf{„Differenz zwischen zwei Zeiträumen anzeigen lassen“}\\
Programm starten $\rightarrow$ Beobachtungszeitraum bearbeiten $\rightarrow$ Startdatum angeben $\rightarrow$ Enddatum angeben $\rightarrow$ “Speichern“ anklicken $\rightarrow$ Zeitliche Differenz auf Weltkarte anzeigen lassen.


\end{itemize}