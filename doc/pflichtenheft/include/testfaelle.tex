% Testfälle  für  die  einzelnen  Produktfunktionen,  die  alle  abgedeckt  sein müssen. Testszenarien für typische Anwendungsszenarien.

\section{Testfälle}

Die Testfälle gliedern sich in allgemeine Testfälle für die GUI und Testfälle für die Datenbank.

	\subsection{GUI Tests}
	\begin{enumerate}[align=left, leftmargin=4em, label={\textbf{\textbackslash T10\arabic*0\textbackslash}} ]
	\item Start des Programms
	\item Kategorie auswählen ($\rightarrow$ \ref{PF:KategorienAuswahl})
	\item Ort auswählen ($\rightarrow$ \ref{PF:OrtAuswahl})
	\item Zusätzlichen Account auswählen ($\rightarrow$ \ref{PF:AccountHinzufügen})
	\item Nach Kategorie suchen ($\rightarrow$ \ref{PF:Suchfunktion})
	\item Nach Ort suchen ($\rightarrow$ \ref{PF:Suchfunktion})
	\item Nach Benutzer suchen ($\rightarrow$ \ref{PF:Suchfunktion})
	\item Weltkarte anzeigen lassen ($\rightarrow$ \ref{PF:Visualisieren})
	\item Daten in Tabelle anzeigen lassen ($\rightarrow$ \ref{PF:AnzeigeDatenblatt})
	\item Auf Karte zoomen ($\rightarrow$ \ref{PF:Zoomen})
	\item Anzeigen von Veränderungen zwischen Zeitpunkten ($\rightarrow$ \ref{PF:Diff})
	\item Beobachtungszeitraum wählen ($\rightarrow$ \ref{PF:WahlZeitraum})
	\item Datenblatt exportieren ($\rightarrow$ \ref{PF:Sicherung}) 
	
	\end{enumerate}
	
	\subsection{Datenbank}
	\begin{enumerate}[align=left, leftmargin=4em, label={\textbf{\textbackslash T20\arabic*0\textbackslash}} ]
		\item Neuen Account hinzufügen ($\rightarrow$ \ref{PF:AccountHinzu})
		\item Kategorie eines Accounts ändern/löschen/hinzufügen ($\rightarrow$ \ref{PF:KategorieAendern})
		\item Ort eines Accounts ändern ($\rightarrow$ \ref{PF:OrtAendern})
	\end{enumerate}

\section{Testszenarien}
\begin{itemize}
	\item \textbf{Programm starten und Twitterdatenströme anzeigen lassen}\\
Programm starten $\rightarrow$ Kategorie auswählen $\rightarrow$ Ort auswählen $\rightarrow$ Zusätzlichen Account auswählen $\rightarrow$ Weltkarte anzeigen lassen $\rightarrow$ Daten in Tabelle anzeigen lassen $\rightarrow$ Auf Karte zoomen. 

	\item \textbf{Bestimmten verifizierten User finden }\\
Programm starten $\rightarrow$ In Suchfeld Accountnamen einfügen $\rightarrow$ Nach gewünschtem Benutzer suchen $\rightarrow$ Falls Account noch nicht vorhanden ist, diesen hinzufügen $\rightarrow$ Ansonsten Benutzer auswählen $\rightarrow$ Weltkarte anzeigen lassen.

	\item \textbf{Bestehenden Account bearbeiten}\\
Programm starten $\rightarrow$ Nach Benutzer suchen $\rightarrow$ Sobald Account gefunden wurde auf „Bearbeiten“ drücken $\rightarrow$ Kategorien und/oder Ortsangaben ändern $\rightarrow$ Auf „Speichern“ klicken.

	\item \textbf{Neuen Account zum Crawlen hinzufügen}\\
Programm starten $\rightarrow$ Auf „Hinzufügen“ klicken $\rightarrow$ Auf „Benutzer“ klicken $\rightarrow$ Accountnamen und ggf. andere Informationen zum Account eingeben $\rightarrow$ Gewünschten Account zum Crawlen auswählen.

	\item \textbf{Differenz zwischen zwei Zeiträumen anzeigen lassen}\\
Programm starten $\rightarrow$ Beobachtungszeitraum bearbeiten $\rightarrow$ Startdatum angeben $\rightarrow$ Enddatum angeben  $\rightarrow$ Zeitliche Differenz auf Weltkarte anzeigen lassen.

\item \textbf{Datenblatt exportieren}\\
Programm starten $\rightarrow$ Kategorie auswählen $\rightarrow$ Ort auswählen $\rightarrow$ Daten in Tabelle anzeigen lassen $\rightarrow$ Datenblatt exportieren.
% Datenblatt exportieren

\end{itemize}