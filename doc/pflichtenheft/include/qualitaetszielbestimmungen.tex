% Anforderungen  an  Stabilität,  Robustheit,  Leistungsfähigkeit  etc.  des Systems.  Dies  beinhaltet  auch  den  Umgang  mit  fehlerhaften  Eingabedaten  oder  fehlerhaften Konfigurationen.
Auch hier wird zwischen Client und Server unterschieden:
\section{Client}
\begin{itemize}
	\item Die Software soll stabil laufen und bei Fehlern nicht ohne Ausgabe einer \textbf{Fehlermeldung} geschlossen werden.
	\item Auch soll mit der in Abschnitt \ref{subsec:hardwareClient} genannter Hard- und Software ausgekommen werden und die Software \textbf{flüssig bedienbar} sein.
	\item Bei \textbf{Verlust der Internetverbindung} bleiben Eingaben des Benutzers erhalten und es wird versucht dessen Anfragen erneut zu senden.
	\item \textbf{Fehlerhafte Eingaben} des Benutzers in Textfeldern führen zu keinem Absturz der Software.
\end{itemize}
\section{Server bzw. Crawler}
\begin{itemize}
	\item Mit der in Abschnitt \ref{subsec:hardwareServer} genannter Hard- und Software soll ausgekommen werden.
	\item Bei Verlust der Internetverbindung oder einem anderen \textbf{Abbruch des Twitter Streams} wird versucht eine neue Verbindung aufzubauen.
	\item \textbf{Nicht lokalisierbare Twitter Nachrichten} sollen das Analyseergebnis nicht verfälschen bzw. es muss nachvollziehbar sein wie viele Retweets nicht lokalisierbar waren.
	\item Kein \textbf{Inhalt einer Twitter Nachricht} (z.B. Autor oder Ort) darf zum Absturz des Crawlers führen.
\end{itemize}
