% vollständige Beschreibung der Aufgabenstellung.

Informationen werden heutzutage hauptsächlich durch Medien verbreitet. %Medien sind heutzutage der hauptsächliche Verbreiter von Informationen. 
Bei Printmedien dauert es jedoch häufig einen ganzen Tag, bis neue Informationen veröffentlicht werden. Plattformen wie Twitter ermöglichen dagegen eine Verbreitung nahezu ohne Verzögerung.\\\\
Mit diesem PSE-Projekt soll ein Werkzeug geschaffen werden, dass eine aufbereitete Darstellung der geografischen Ausbreitung thematisch kategorisierter Tweets und Retweets ermöglicht.

%kleiner Hack um Fußnote in die Überschrift zu bekommen, sie aber aus dem Inhaltsverzeichnis rauszuholen
\section[Aufgabenstellung - Analyse der Informationsverbreitung]{\texorpdfstring{Aufgabenstellung - Analyse der Informationsverbreitung\protect\footnote{\url{http://dsn.tm.kit.edu/lehre_3581.php}}}}
"`Big Data"' ist nach wie vor in aller Munde. Öffentliche wie nicht-öffentliche Inhalte des World Wide Web werden beispielsweise von Unternehmen ausgewertet, um Rückschlüsse zu ziehen, die für zielgerichtete Werbung, zur Marktforschung oder zur Verbesserung von Diensten genutzt werden können. Durch die Verbreitung von Online Social Networks, deren Inhalte in großen Teilen öffentlich zugänglich sind, können Inhalte auch durch Dritte gesammelt und analysiert werden. Beispielsweise ist es möglich, die Beliebtheit einzelner Inhalte und ihre geographische Verbreitung zu beobachten.\\\\
Im PSE-Projekt "`Visualizing Trends - Was verrät uns Twitter?"' sollen die Möglichkeiten zu einer Analyse großer Mengen von Inhalten aus dem Twitter-Netzwerk untersucht werden. Ziel ist es, ein Werkzeug zur Sammlung, Auswertung und Visualisierung von Inhalten des Twitter-Netzwerks zu implementieren.\\\\
\emph{Team B "`Analyse der Informationsverbreitung"'} untersucht die Datenflüsse über geographische Grenzen hinweg, beantwortet also die Frage "`Wohin verbreiten sich welche Inhalte?"'. Hierfür wird unter anderem auf die Tweet-Retweet-Beziehung zwischen Nachrichten zurückgegriffen.\\\\
Kern des Projekts ist die Umsetzung einer effizienten Infrastruktur, die die folgenden Schritte durchführt:
\begin{itemize}
	\item Sammeln von Daten mittels der Twitter-API
	\item Auswertung durch austauschbare Analysekomponenten
	\item Visualisierung der Analyse-Ergebnisse in einer graphischen Oberfläche
\end{itemize}
Teile der Analyse der Twitter-Nachrichten können mittels existierender Werkzeuge zum Beispiel zur Inhaltsbestimmung und zur geographischen Zuordnung gelöst werden.\\\\
%Neben dem Entwurf und der Implementierung eines IT-Systems im Team können in dem Projekt Erfahrungen mit der Interaktion zwischen Desktop-Applikationen und Web-Diensten sowie mit Fragen der Effizienz bei der Verarbeitung großer Datenmengen gesammelt werden.
