\section{Aufbau}
\subsection{Paketdiagramm}

\begin{figure}[h!]
	\centering
	\includegraphics[width=\textwidth,height=\textheight,keepaspectratio=true]{dia/GUIPackage}
	\caption{Paketdiagramm der GUI}
	\label{fig:GUI}
\end{figure}

\subsection{Klassendiagramm}
Die GUI ermöglicht eine Interaktion des Benutzers mit der Anwendung und stellt die über den Crawler gesammelten Daten grafisch aufbereitet dar.\\

\begin{figure}[h!]
	\centering
	\includegraphics[width=\textwidth,height=\textheight,keepaspectratio=true]{dia/TwitterGUI_Erweiterung}
	%\includegraphics[width=\textheight,height=\textwidth,keepaspectratio=true,angle=-90]{dia/TwitterGUI_Erweiterung}
	\caption{Klassendiagramm der GUI}
	\label{fig:GUI}
\end{figure}
\begin{description}
	\item[GuiController] Diese Klasse enthält alle GUI-Elemente. Damit kann über diese Klasse jedes einzelne Element angesprochen und damit gesteuert werden. Außerdem speichert sie die jeweils aktuellen Resultate der Datenbankabfragen zentral. Jede Erweiterung muss sich im GuiController als 'Observer' eintragen, um über Änderungen der Daten informiert zu werden. 
	\item[GuiElement] Interface, das jedes GUI-Element implementieren muss.
	\item [selectionOfQuery] Dieses Paket enthält Darstellung und Anwendungslogik für die Auswahl einer Suchanfrage (Auswahl von Kategorie, Land, usw.)
	\item[databaseOptions] Dieses Paket enthält die Darstellung und Anwendungslogik für Änderungen an der Datenbank, wie das Hinzufügen eines bisher nicht mitgetrackten Accounts.
	
	\item [standardMap] Das Paket enthält Anwendungslogik und Darstellung für die Standardkarte, welche die Länder nach dem jeweiligen Tweet-Retweet-Aufkommen einfärbt.
	\item [table] Paket, welches Anwendungslogik und Darstellung für die Erstellungen und Anzeige des Datenblattes zur aktuellen Anfrage enthält.
	\item [timeSliderMap] Paket, welches Anwendungslogik und Darstellung für Erstellung und Anzeige des Tweet-Retweet-Aufkommens in Abhängigkeit des gewählten Zeitraums anzeigt.
	\item [myUnfoldingMap] Diese Klasse kapselt die eigentliche Darstellung sämtlicher Kartenanzeigen. Sie ist die 'Schnittstelle' zur Unfolding-Library, welche für die Anzeige der Weltkarte verwendet wird.
\end{description}
\section{Sequenzdiagramme}
\subsection{GUI Initialisierung}

\cref{fig:GUIStartSeq} veranschaulicht den Ablauf der Initialisierung der Benutzeroberfläche. Nach dem Aufrufen des Programmes durch den Benutzer erzeugt die Main-Klasse zuerst mit \emph{DBIgui} eine Verbindung zu Datenbank.
Daraufhin wird mittels eines von JavaFX bereitgestellten FXML-Loaders die Hauptverwaltungseinheit der GUI, bestehend aus ihrer Anzeige, der \emph{GuiControllerView} und der zugehörigen Kontolle, dem \emph{GuiController} geladen und initialisiert.
Danach können GUI-Elemente in beliebiger Reihenfolge hinzugefügt werden, wie hier am Beispiel der \emph{selectionOfQuery}.
Auch hier erstellt der FXML-Loader Anzeige und Kontrolle des GUI-Elements und ruft dann die zugehörige initialize-Methode auf.
In dieser verbindet sich das GUI-Element mit dem \emph{GuiController} und erhält über diesen eine Verbindung zur Datenbank. Es holt sich die nötigen Daten ab und initialisiert dann seine Komponenten und erstellt die notwendigen Event-Handler.
Nach erfolgreicher Erstellung des GUI-Elements meldet die  Main-Klasse die Kontrolle des GUI-Elements beim \emph{GuiController} als Beobachter an.
Der Vorgang ist beendet, sobald alle GUI-Elemente an der Hauptverwaltungseinheit angemeldet sind.

\begin{figure}[h!]
	\centering
	\includegraphics[width=\textwidth,height=0.5\textheight,keepaspectratio=true]{dia/GUISequenz_Start}
	\caption{Sequenzdiagramm der Initialisierung der GUI.}
	\label{fig:GUIStartSeq}
\end{figure}
\subsection{Kommunikation: GUI und Datenbank}
In \cref{fig:GUISeq} ist die Auswahl einer neuen Kategorie in einem Sequenzdiagramm dargestellt. Der Benutzer klickt auf eine Kategorie in der Liste, woraufhin \emph{handle(mouseEvent)} ausgelöst wird. Der \emph{SelectionOfQuerryController} gibt dieses Ereignis dem \emph{GuiController} weiter bzw. übergibt ihm eine Liste aller IDs ausgewählter Kategorien. Diese Klasse wiederum läd mittels \emph{getData} neue Daten zu den sich geänderten Kategorien. Die IDs ausgewählter Kategorien, Locations und Accounts (siehe Parameter der Operation \emph{getData}) sind dabei lokal in der Klasse \emph{GUIController} gespeichert.

Alle GUI-Elemente werden dann mittels \emph{update} über geänderte Daten informiert. Als Parameter wird ein Enum übergeben, der den Typ der Datenänderung angibt. Es gibt folgende Typen:
\begin{description}
	\item[TWEET] Die Information Anzahl Retweets pro Land und Tweet hat sich geändert. Hier ist bspw. eine Karte, die die Anzahl der Retweets pro Land graphisch darstellt, oder eine Tabelle mit vorher genannten Daten betroffen. 
	\item[CATEGORY] Eine lokale Liste der Kategorien wurde verändert. Die Liste, aus der der Benutzer Kategorien auswählt muss aktualisiert werden.
	\item[LOCATION] Länder bzw. Orte wurden aktualisiert. Auch hier ist z.B. die Liste im Filter, aus dem der Benutzer ein Land auswählt betroffen.
\end{description}
Bei Änderungen vom Typ \emph{TWEET} ist bspw. die \emph{StandardMapView} betroffen und fordert mittels \emph{getTweets()} die neue Informationen an und aktualisiert somit die aktuell angezeigte Karte. Einfachheitshalber wird hier nur die Aktualisierung eines GUI Elements visualisiert. Die Operation \emph{udpate} wird auf jedem GUI-Element, welches sich beim \emph{GUIController} mittels \emph{subscribe} angemeldet hat, aufgerufen.
\begin{figure}[h!]
	\centering
	\includegraphics[width=\textwidth,height=\textheight,keepaspectratio=true]{dia/TwitterGUI_Erweiterung_SequenzDiagramm}
	\caption{Sequenzdiagramm für Auswahl einer neuen Kategorie in der GUI.}
	\label{fig:GUISeq}
\end{figure}
