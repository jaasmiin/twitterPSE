\section{Aufbau}
Die GUI ermöglicht eine Interaktion des Benutzers mit der Anwendung und stellt die über den Crawler gesammelten Daten grafisch aufbereitet dar.\\
\subsection{Paketdiagramm}
Das Paket {guiControll} beinhaltet die Steuerlogik der GUI. Die einzelnen Unterpakete enthalten  Komponenten der GUI. Somit wird innerhalb der Unterpakete ein starker Zusammenhalt erreicht. Gleichzeitig kann die GUI einfach um Funktionalität erweitert werden, indem weitere Komponenten hinzugefügt werden. Da die Unterpakete keine Beziehungen oder Abhängigkeiten untereinander aufweisen, kann somit eine Erweiterung der Funktionalität ohne Änderung der Programmlogik der anderen Pakete erfolgen. 


\begin{figure}[h!]
	\centering
	\includegraphics[width=\textwidth,height=\textheight,keepaspectratio=true]{dia/GUIPackage}
	\caption{Paketdiagramm der GUI}
	\label{fig:GUI}
\end{figure}


Im Folgenden ist eine kurze Beschreibung der einzelnen Unterpakete zu finden:
\begin{description}
		\item [selectionOfQuery] Darstellung und Anwendungslogik  der Auswahl von Suchanfragen (Auswahl von Kategorie, Land, usw.) werden in diesem Paket gekapselt.
		\item[databaseOptions] Dieses Paket enthält die Darstellung und Anwendungslogik für Änderungen an der Datenbank, wie das Hinzufügen eines bisher nicht mitgetrackten Accounts.
		\item[search] Dieses Paket enthält Darstellung und Anwendungslogik für eine allgemeine Suchfunktion.
		\item [standardMap] Das Paket enthält Anwendungslogik und Darstellung für die Standardkarte, welche die Länder nach dem jeweiligen Tweet-Retweet-Aufkommen einfärbt.
		\item [table] Paket, welches Anwendungslogik und Darstellung für die Erstellungen und Anzeige des Datenblattes zur aktuellen Anfrage enthält.
		\item[diffMap] Paket, welches Anwendungslogik und Darstellung für die Anzeige der Darstellung des Unterschiedes im Tweet-Retweet-Verhalten zwischen zwei Zeitpunkten enthält.
		\item [timeSliderMap] Paket, welches Anwendungslogik und Darstellung für Erstellung und Anzeige des Tweet-Retweet-Aufkommens in Abhängigkeit des gewählten Zeitraums anzeigt.
		\item [unfolding] In diesem Paket wird die Kartendarstellung berechnet. Dabei wird die unfolding-Bibliothek verwendet.
		
\end{description}

\subsection{Klassendiagramm}
Die Klassen werden soweit möglich nach Unterpaketen gegliedert beschrieben. Die jeweiligen 'View' Klassen werden nicht einzeln beschrieben, da sie jeweils nur die grafische Darstellung der einzelnen Komponenten kapseln.

\begin{figure}[h!]
	\centering
	\includegraphics[width=\textwidth,height=\textheight,keepaspectratio=true]{dia/TwitterGUI_Erweiterung}
	%\includegraphics[width=\textheight,height=\textwidth,keepaspectratio=true,angle=-90]{dia/TwitterGUI_Erweiterung}
	\caption{Klassendiagramm der GUI}
	\label{fig:GUI}
\end{figure}
\quad
\begin{description}
	
	\item[GuiController] Diese Klasse enthält alle GUI-Elemente. Damit kann über diese Klasse jedes einzelne Element angesprochen werden. Außerdem speichert sie die jeweils aktuellen Daten der Datenbankabfragen zentral. Jede Erweiterung muss sich im GuiController als 'Observer' eintragen, um über Änderungen der Daten informiert zu werden. 
	\item[GuiElement] Abstrakte Klasse, von welchem jedes GUI-Element erben muss. Enthält die Rückverbindung zum 'GuiController'.
	\item[selectionOfQuery]  
	\quad
		 \begin{description}  	
		\item[SelectionSuperController] Steuert die zugehörige 'View'. Die eigentliche Steuerung wird an die Subcontroller 'SelectionController' und 'SearchController' delegiert.
		\item[SelectionController] Enthält die Programmlogik für den Bereich der Auswahl der gewünschten Kategorien, Länder und Accounts.
		\item[SearchController] Enthält die Programmlogik für die Suche nach einzelnen Kategorien, Ländern und Accounts in der Datenbank, welche ausgewählt werden können. 
		
		\end{description}
	\item[databaseOptions] 	\quad
		\begin{description}
			\item[DatabaseOptController] Enthält die Progammlogik der zugehörigen 'View'-Klasse.
		\end{description}
	\item[search] \quad
		\begin{description}
			\item[SearchControl] Enthält die Methoden, um eine allgemeine Suche durchzuführen. Dabei wird die Datenbank nach passenden Resultaten durchsucht. Die zugehörige 'View'-Klasse kapselt die Darstellung.
		\end{description}
		
    \item[standardMap] \quad
	    \begin{description}
	    	\item [StandardMapController] Steuert und berechnet die Anzeige der ausgewählten Daten auf der Weltkarte.
	    \end{description}
	\item[diffMap] \quad
		\begin{description}
			\item [DiffMapController] Berechnet die Differenz im Tweet-Retweet-Verhalten zwischen zwei Zeitpunkten und steuert deren Visualisierung in der 'View'-Klasse.
		\end{description}
    \item[table] \quad
	    \begin{description}
	    	\item [ContentTableController] Steuert und berechnet die Anzeige der Tabellendarstellung der Daten.
	    \end{description}
    \item[timesliderMap] \quad
	    \begin{description}
	    	\item [TimeSliderMapController] Steuert und berechnet die Anzeige auf der Weltkarte. Zusätzlich stellt die Klasse eine Methode bereit, über welche der Zeitraum der dargestellten Daten gesetzt werden kann.
	    \end{description}
	\item[unfolding] \quad
		\begin{description}
			\item [MyUnfoldingMap] Diese Klasse kapselt die eigentliche Darstellung sämtlicher Kartenanzeigen. Sie ist die 'Schnittstelle' zur Unfolding-Library, welche für die Anzeige der Weltkarte verwendet wird.
		\end{description}
	
\end{description}
\section{GUI Initialisierung}

\cref{fig:GUIStartSeq} veranschaulicht den Ablauf der Initialisierung der Benutzeroberfläche. Nach dem Aufrufen des Programmes durch den Benutzer ruft die main-Methode der GuiController-Klasse zuerst, wie von JavaFX vorgegeben die launch- und dann darin die start-Methode auf.
In dieser wird als erstes ein Singleton des \emph{GuiControllers} und eine Verbindung zur Datenbank mit einer Instanz von \emph{DBIgui} erstellt. Danach wird mittels des von JavaFX bereitgestellten FXML-Loaders die Anzeige der GUI, die \emph{GuiView} und dann die zugehörigen GUI-Elemente geladen.
Exemplarisch für diesen Vorgang wird das Element \emph{SelectionOfQuery} herausgegriffen.
Der FXML-Loader erstellt mit \emph{SelectionOfQueryView} und \emph{SelectionOfQueryController} Anzeige und Kontrolle des GUI-Elements und ruft die initialize-Methode des \emph{SelectionOfQueryController} auf.
Darin holt sich die Kontrolle des GUI-Elements durch die getInstance-Methode eine Referenz auf den \emph{GuiController}, meldet sich bei diesem mit der subscribe-Methode an und setzt die benötigten Event-Handler.
Die Daten aus der Datenbank werden nun wie im folgenden Kapitel geladen und dann in der GUI dargestellt.
\begin{figure}[h!]
	\centering
	\includegraphics[width=\textwidth,height=\textheight,keepaspectratio=true]{dia/GUISequenz_Start}
	\caption{Sequenzdiagramm der Initialisierung der GUI.}
	\label{fig:GUIStartSeq}
\end{figure}
\section{Kommunikation: GUI und Datenbank}
In \cref{fig:GUISeq} ist die Auswahl einer neuen Kategorie in einem Sequenzdiagramm dargestellt. Der Benutzer klickt auf eine Kategorie in der Liste, woraufhin \emph{handle(mouseEvent)} ausgelöst wird. Der \emph{SelectionOfQuerryController} gibt dieses Ereignis dem \emph{GuiController} weiter bzw.  teilt ihm mit welche Kategorie ausgewählt wurde. Diese Klasse wiederum lädt mittels \emph{getSumOfData} neue Daten zu den sich geänderten ausgewählten Kategorien. Die IDs ausgewählter Kategorien, Locations und Accounts (siehe Parameter der Operation \emph{getSumOfData}) sind dabei lokal in der Klasse \emph{GUIController} gespeichert.

Alle GUI-Elemente werden dann mittels \emph{update} über geänderte Daten informiert. Als Parameter wird ein Enum übergeben, der den Typ der Datenänderung angibt. Es gibt folgende Typen:
\begin{description}
	\item[TWEET] Die Information Anzahl Retweets pro Land und Tweet hat sich geändert. Hier ist bspw. eine Karte, die die Anzahl der Retweets pro Land grafisch darstellt, oder eine Tabelle mit vorher genannten Daten betroffen. 
	\item[CATEGORY] Die lokale Liste der Kategorien wurde geändert. Die Liste, aus der der Benutzer Kategorien auswählt muss aktualisiert werden.
	\item[LOCATION] Länder bzw. Orte wurden aktualisiert. Auch hier ist z.B. die Liste im Filter, aus dem der Benutzer ein Land auswählt betroffen.
	\item[ACCOUNT] Accounts wurden verändert, es muss z.B. die Account Liste im Filter aktualisiert werden.
\end{description}
Bei Änderungen vom Typ \emph{TWEET} ist beispielsweise die \emph{StandardMapView} betroffen und fordert mittels \emph{getSummedData()} die neue Informationen an und aktualisiert somit die aktuell angezeigte Karte. Einfachheitshalber wird hier nur die Aktualisierung eines GUI Elements visualisiert. Die Operation \emph{udpate} wird auf jedem GUI-Element, welches sich beim \emph{GUIController} mittels \emph{subscribe} angemeldet hat, aufgerufen.
\begin{figure}[h!]
	\centering
	\includegraphics[width=\textwidth,height=\textheight,keepaspectratio=true]{dia/TwitterGUI_Erweiterung_SequenzDiagramm}
	\caption{Sequenzdiagramm für Auswahl einer neuen Kategorie in der GUI.}
	\label{fig:GUISeq}
\end{figure}
