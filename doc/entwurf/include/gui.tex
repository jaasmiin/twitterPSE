\section{Aufbau}

\subsection{Paketdiagramm}

\subsection{Klassendiagramm}
Die GUI ermöglicht die Interaktion des Benutzers mit der Anwendung und stellt die über den Crawler gesammelten Daten grafisch aufbereitet dar. 
\begin{figure}[h!]
	\centering
	\includegraphics[width=\textheight,height=\textwidth,keepaspectratio=true,angle=-90]{dia/TwitterGUI_Erweiterung}
	\caption{Klassendiagramm der GUI}
	\label{fig:GUI}
\end{figure}
\begin{description}
	\item[GuiController] Diese Klasse enthält alle GUI-Elemente. Damit kann über diese Klasse jedes einzelne Element angesprochen und damit gesteuert werden. Außerdem speichert sie die jeweils aktuellen Resultate der Datenbankabfragen zentral. Jede Erweiterung muss sich im GuiController als 'Observer' eintragen, um über Änderungen der Daten informiert zu werden. 
	\item[GuiElement] Interface, das jedes GUI-Element implementieren muss.
	\item [selectionOfQuery] Dieses Paket enthält Darstellung und Anwendungslogik für die Auswahl einer Suchanfrage (Auswahl von Kategorie, Land, usw.)
	\item[databaseOptions] Dieses Paket enthält die Darstellung und Anwendungslogik für Änderungen an der Datenbank, wie das Hinzufügen eines bisher nicht mitgetrackten Accounts.
	
	\item [standardMap] Das Paket enthält Anwendungslogik und Darstellung für die Standardkarte, welche die Länder nach dem jeweiligen Tweet-Retweet-Aufkommen einfärbt.
	\item [table] Paket, welches Anwendungslogik und Darstellung für die Erstellungen und Anzeige des Datenblattes zur aktuellen Anfrage enthält.
	\item [timeSliderMap] Paket, welches Anwendungslogik und Darstellung für Erstellung und Anzeige des Tweet-Retweet-Aufkommens in Abhängigkeit des gewählten Zeitraums anzeigt.
	\item [myUnfoldingMap] Diese Klasse kapselt die eigentliche Darstellung sämtlicher Kartenanzeigen. Sie ist die 'Schnittstelle' zur Unfolding-Library, welche für die Anzeige der Weltkarte verwendet wird.
\end{description}

\section{Sequenzdiagramme}
\begin{figure}[h!]
	\centering
	\includegraphics[width=\textwidth,height=0.6\textheight,keepaspectratio=true]{dia/GUISequenz_Start}
	\caption{Sequenzdiagramm der Initialisierung der GUI.}
	\label{fig:GUIStartSeq}
\end{figure}
\begin{figure}[h!]
	\centering
	\includegraphics[width=\textheight,height=\textwidth,keepaspectratio=true,angle=-90]{dia/TwitterGUI_Erweiterung_SequenzDiagramm}
	\caption{Sequenzdiagramm für Auswahl einer neuen Kategorie in der GUI.}
	\label{fig:GUISeq}
\end{figure}
