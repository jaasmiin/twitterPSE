\section{Automatisierte Tests}

\begin{table}[h]
\begin{tabular} {lcr}
Bereich (Klasse / Package) & Überdeckung \\
	\textbf{mysql} &  \\
		\hspace*{3mm}AccessData & 100\% \\
		\hspace*{3mm}DBConnection & 69\% \\
		\hspace*{3mm}DBcategorizer & 68\% \\
		\hspace*{3mm}DBcrawler & 79\% \\
		\hspace*{3mm}DBgui & 76\% \\
	\textbf{mysql.util} &  \\
		\hspace*{3mm}LoggerUtil & 88\% \\
		\hspace*{3mm}Util & 88\% \\
	\textbf{mysql.result} &  \\
		\hspace*{3mm}Account & 99\% \\
		\hspace*{3mm}Category & 99\% \\
		\hspace*{3mm}Location & 92\% \\
		\hspace*{3mm}Result & 100\% \\
		\hspace*{3mm}Retweets & 100\% \\
		\hspace*{3mm}Tweets & 100\% \\
		\hspace*{3mm}TweetsAndRetweets & 100\% \\
	\end{tabular}
	\caption{Übersicht über die Testüberdeckung}
\end{table}

\section{Manuelle Tests}

\subsection{Bestehenden Account bearbeiten}

\subsubsection{Kategorie zu einem Account hinzufügen}
"`Data"' in der Menüleiste auswählen, zu Unterpunkt "`Add category"' wechseln $\to$ Account suchen,  z.B. \textit{KATY PERRY} $\to$ Neue Kategorie aus Kategorienbaum auswählen, z.B. \textit{Top/Arts/Music}. Durch Doppelklick hinzufügen. $\to$ Am Ende "`Finish"' drücken, um Vorgang abzuschließen.

\subsubsection{Ort ändern/hinzufügen}
"`Data"' in der Menüleiste auswählen, zu Unterpunkt "`Add/Change location "' wechseln $\to$ Account suchen,  z.B. \textit{KATY PERRY} $\to$ Das aktuelle Land und eine Liste aller Länder werden angezeigt. $\to$ Wähle neues Land aus, z.B. \textit{United States of America} $\to$ Drücke "`Change"'

\subsection{Export}
Um die Exportfunktion zu testen, wurde folgende zwei Tests durchgeführt:
\begin{itemize}
\item Auswahl einer Kategorie (hier: \textit{World}) und eines Landes (hier: \textit{United States of America}). Nun wurde über "`File"' $\to$ "`Export..."' eine .csv Datei als Zieldatei ausgewählt und die Daten wurden exportiert.
\item Auswahl von zwei Kategorien (hier: \textit{Software}, \textit{Technology}), zweier Länder (hier: \textit{Germany}, \textit{France}) und zusätzlich eines Accounts (hier: \textit{Qualcomm}). Nun wurde über "`File"' $\to$ "`Export"'... eine .csv Datei als Zieldatei ausgewählt und die Daten wurden exportiert.
\end{itemize}
Bei beiden Exporten wurde die exportierte Datei eingehend untersucht. Dabei wurden keine Abweichungen von der Spezifikation gefunden. Auch stimmen die exportierten Werte mit den original-Werten der Anwendung überein.
