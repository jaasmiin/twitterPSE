\section{Automatisierte Tests}
Die Tests wurden in \emph{Eclipse} mittels \emph{JUnit} durchgeführt und die Überdeckung mit \emph{Eclemma} berechnet.
In folgenden Tabellen ist die Überdeckung für das Paket \emph{mysql} (s.S. \pageref{tbl:coverageMysql}) bzw. \emph{gui}
(s.S. \pageref{tbl:coverageGui}) zu finden.
\begin{table}[h]
	\centering
	\begin{tabular} {l||r}
		\label{tbl:coverageMysql}
		Bereich (Klasse / Package) & Überdeckung \\
		\hline
		\hline
		\textbf{mysql} & \\
		\hspace*{3mm}AccessData & 100\% \\
		\hspace*{3mm}DBConnection & 69\% \\
		\hspace*{3mm}DBcategorizer & 68\% \\
		\hspace*{3mm}DBcrawler & 79\% \\
		\hspace*{3mm}DBgui & 76\% \\
		\hline
		\textbf{mysql.util} & \\
		\hspace*{3mm}LoggerUtil & 88\% \\
		\hspace*{3mm}Util & 88\% \\
		\hline
		\textbf{mysql.result} & \\
		\hspace*{3mm}Account & 99\% \\
		\hspace*{3mm}Category & 99\% \\
		\hspace*{3mm}Location & 92\% \\
		\hspace*{3mm}Result & 100\% \\
		\hspace*{3mm}Retweets & 100\% \\
		\hspace*{3mm}Tweets & 100\% \\
		\hspace*{3mm}TweetsAndRetweets & 100\% \\
	\end{tabular}
	\caption{Übersicht über die Testüberdeckung des mysql Pakets}
\end{table}
\begin{table}[h]
	\centering
	\begin{tabular} {l||r}
		\label{tbl:coverageGui}
		Bereich (Klasse / Package) & Überdeckung \\
		\hline
		\hline
		\textbf{gui} \\
		\hspace*{3mm}GUIController & 67\% \\
		\hspace*{3mm}SelectionHashList & 58\% \\
		\hspace*{3mm}GUIElement & 97\% \\
		\hspace*{3mm}InfoRunnable & 96\% \\
		\hspace*{3mm}Labels & 91\% \\
		\hspace*{3mm}Util & 92\% \\
		\hspace*{3mm}InputElement & 100\% \\
		\hspace*{3mm}OutputElement & 100\% \\
		\hspace*{3mm}PPPPRunnable & 100\% \\
		\hspace*{3mm}PPRunnable & 100\% \\
		\hspace*{3mm}PRunnable & 100\% \\
		\hline
		\textbf{gui.selectionOfQuery} \\
		\hspace*{3mm}SelectionofQueryController & 54\% \\
		\hspace*{3mm}SelectionofQuerySelectedController & 56\% \\
		\hline
		\textbf{gui.unfolding} \\
		\hspace*{3mm}MyUnfoldingMap & 51\% \\
		\hspace*{3mm}MyDataEntry & 52\% \\
		\hline
		\textbf{gui.table} \\
		\hspace*{3mm}ContentTableController & 100\% \\
		\hspace*{3mm}InternAccount & 56\% \\
	\end{tabular}
	\caption{Übersicht über die Testüberdeckung der GUI}
\end{table}
\section{Testszenarien}

In einer GUI können bestimmte Szenarien nur schwer automatisiert getestet werden, deswegen wurden einige Tests manuell ausgeführt.

\subsection{Programm starten und Twitterdatenströme anzeigen lassen}
Programm starten $\to$ auf "`Category"' klicken $\to$ eine Kategorie auswählen, z.B. \textit{Software} und durch Doppelklick hinzufügen $\to$ auf "`Location"' klicken und ein Land auswählen, z.B. \textit{United States of America} und durch Doppelklick hinzufügen. $\to$ auf "`Account"' klicken und einen Account auswählen, z.B. \textit{Bill Gates} und durch Doppelklick hinzufügen. $\to$ "`Table"' anklicken und Tabelle anzeigen lassen $\to$ auf die Karte klicken und mit den Tasten "`+"' und "`-"' zoomen.
Die Daten werden angezeigt und aktualisieren sich sowohl in der Tabelle als auch in der Karte nach jeder Änderung der Kombination von Accounts, Orten und Kategorien.

\subsection{Bestehenden Account bearbeiten}
\subsubsection{Kategorie zu einem Account hinzufügen}
"`Database"' in der Menüleiste auswählen, zu Unterpunkt "`Add category"' wechseln $\to$ Account suchen, z.B. \textit{KATY PERRY} $\to$ Neue Kategorie aus Kategorienbaum auswählen, z.B. \textit{Top/Arts/Music}. Durch Doppelklick hinzufügen. $\to$ Am Ende "`Finish"' drücken, um Vorgang abzuschließen.
\subsubsection{Ort ändern/hinzufügen}
"`Database"' in der Menüleiste auswählen, zu Unterpunkt "`Edit location "' wechseln $\to$ Account suchen, z.B. \textit{KATY PERRY} $\to$ Das aktuelle Land und eine Liste aller Länder werden angezeigt. $\to$ Wähle neues Land aus, z.B. \textit{United States of America} $\to$ Drücke "`Change"'
\subsection{Neuen Account zum Crawlen hinzufügen}
"`Database"' in der Menüleiste auswählen, zu Unterpunkt "`Add account "' wechseln $\to$ Namen in Suchfeld eingeben und mit Klicken auf den "`Search-Button "' bestätigen $\to $ gefundene Accounts werden angezeigt $\to$ gewünschten Account durch auswählen und durch Klicken auf den "`Add-Button"' zur Datenbank/zum Crawler hinzufügen $\to$ Menüfenster durch Klicken auf "`Close"' schließen.
Dieses Szenario wurde wie im Pflichtenheft beschrieben implementiert.

\subsection{Datenblatt exportieren}
Um die Exportfunktion zu testen, wurde folgende zwei Tests durchgeführt:
\begin{itemize}
	\item Auswahl einer Kategorie (hier: \textit{World}) und eines Landes (hier: \textit{United States of America}). Nun wurde über "`File"' $\to$ "`Export..."' eine .csv Datei als Zieldatei ausgewählt und die Daten wurden exportiert.
	\item Auswahl von zwei Kategorien (hier: \textit{Software}, \textit{Technology}), zweier Länder (hier: \textit{Germany}, \textit{France}) und zusätzlich eines Accounts (hier: \textit{Qualcomm}). Nun wurde über "`File"' $\to$ "`Export"'... eine .csv Datei als Zieldatei ausgewählt und die Daten wurden exportiert.
\end{itemize}
Bei beiden Exporten wurde die exportierte Datei eingehend untersucht. Dabei wurden keine Abweichungen von der Spezifikation gefunden. Auch stimmen die exportierten Werte mit den original-Werten der Anwendung überein.



