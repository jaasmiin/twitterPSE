\section{Tests}

Im Folgenden werden die Tests beschrieben, welche durchgeführt wurden. 

\subsection{GUI}

Im Bereich der GUI wurden einige JUnit-Testfälle geschrieben, die hauptsächlich den Bereich des \emph{GUIControllers} betreffen:
\begin{description}
	\item[testGetCategories, testGetAccounts] Lädt Kategorien bzw. Accounts aus der Datenbank.
	\item[testSelectAccounts] Wählt einen Account\footnotemark[1] für den nächsten Query aus.
	\item[testDeselectAccounts] Entfernt einen Account\footnotemark[1] aus dem Query.
	\item[testGetDataByLocation] Liefert alle geladenen Retweets nach Ländern sortiert.
	\item[testSubscribe] Anmelden beim \emph{GUIController}
\footnotetext[1]{Diese Testmethode existiert auch für Orte und Kategorien.}
\end{description}

Diese Testfälle decken die Basisfunktionalität des \emph{GUIControllers} ab. In den Unterpaketen des Pakets \emph{gui} wurde größtenteils von Hand getestet, da  hierbei hauptsächlich die richtige Behandlung von \emph{Events} getestet werden muss, hierbei allerdings automatisierte Test schwer umzusetzen sind.


\subsection{Crawler}

Der Crawler wurde nicht mit automatisierten JUnit-Tests geprüft, stattdessen wurde hier sehr intensiv mittels Mitprotokollieren des Progammablaufs auf dem Server gearbeitet. Da hier das Hauptproblem vor allem die Effizienz war.

\subsection{mysql}

In diesem Paket wurde sehr viel mit JUnit-Test gearbeitet, diese werden im Folgenden exemplarisch dargestellt. Alle Tests wurden auf einer Datenbank mit Testdaten durchgeführt.
\subsubsection{DBcategorizer}
\quad
\begin{description}
	\item[testGetNonCategorized] Liefer nicht kategorisierte Accounts zurück.
	\item[testAddCategoryToAccount] Füge Kategorie zu Account hinzu.
	\item[...]
\end{description}
\subsubsection{DBcrawler}
\begin{description}
	\item[testGetNonVerifiedAccounts] Fragt nicht verifizierten Accounts an.
	\item[testAddDay] Fügt Datum hinzu.
	\item[testGetAccounts] Fragt Accounts aus Datenbank an.
	\item[testAddAccount] Fügt Account hinzu.
	\item[...]
	\item[testAddRetweet] Fügt Retweet hinzu.
	\item[...]	
\end{description}
\subsubsection{DBgui}
\begin{description}
	\item [testGetAccounts] Lädt Accounts aus der Datenbank.
	\item [...]
\end{description}
\quad



